\section{Conclusion}

Testing smart grids, and IoT in general, in the cloud is popular and tightly coupled. Smart grid systems' complexity and heterogenity demand for services where traditional technologies reach their limits but the advantages of cloud technologies, like elasticity, costs per usage, efficiency, flexibility, or reduced maintenance effort arise. Cloud computing models, like \textit{platform as a service} and \textit{software as a service} facilitate the life of engineers or researchers by providing complete development and deployment platforms resp. taking responsibility on a software's entire lifecycle.

In general there is consensus that smart grid testing is crucial in view of the criticality and vulnerability of electricity supply systems, but also that testing smart grids, especially from an end-to-end perspective is still a big challenge. The following prominent approaches for doubling / replicating systems or system components for testing purposes have be identified: (Co-)simulations, testbeds, and digital twins. They can be employed in a complementary manner in one and the same framework. This review took a closer look at concrete examples where each of the three are employed. In the end, all of them head towards the same goal. They attempt to provide an as holistic and flexible test environment as possible by keeping computational costs and maintenance effort within an acceptable limit.
