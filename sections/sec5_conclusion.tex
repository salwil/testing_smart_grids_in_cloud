\section{Conclusion}

Testing smart grids, and IoT in general, in the cloud is popular and the two things are tightly interconnected. Smart grid systems' complexity and heterogenity demand for services where traditional technologies reach their limits but the advantages of cloud technologies, like elasticity, costs per usage, efficiency, flexibility, or reduced maintenance effort arise. Cloud computing models, like \textit{platform as a service} and \textit{software as a service} facilitate the life of engineers or researchers by providing complete development and deployment platforms resp. by taking responsibility on a software's entire lifecycle.

In general there is consensus that smart grid testing is crucial in view of the criticality and vulnerability of electricity supply systems, but also that testing smart grids, especially from an end-to-end perspective is still a big challenge. Three concrete examples of how systems or system components can be replicated for testing purposes have be presented. They build up on (co-)simulations, testbeds, or a digital twin. At the core, all of them head towards the same goal and face similar obstacles. They attempt to provide an as holistic and flexible test environment as possible by keeping computational and economical costs and maintenance effort within an acceptable limit.
