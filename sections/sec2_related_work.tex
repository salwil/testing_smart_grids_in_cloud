\section{Related Work (draft!)}
So far no literature review covering cloud testing techniques in smart grid industry have been found. However research has been done on the topic itself in the past years. Many authors investigated on IoT cloud computing and point out its potential to improve efficiency and reduce costs in IoT because computing resources can be flexibly scaled and virtualized.

However, in the context of smart grids, hybrid-cloud usage seems to be clearly favoured over "cloud-only" approaches for several reasons. \citeauthor{talaat2020hybrid} for example suggest to integrate data processing hardware devices to a private cloud to overcome security issues in the public cloud. \citeauthor{zahoor2018cloudmanag} recommend cloud-fog-based smart grid model for efficient resource management \cite{zahoor2018cloudmanag} and utilization \cite{zahoor2018cloudutil}. Fog computing solves problems of latency, delay, response time, and requests per hour \cite{zahoor2018cloudmanag}. The amount of transmitted data and data transmission time can be significantly reduced because the \textit{fog approach reduces the amounts of data transmitted to the cloud by performing data processing closer to the place where data are being produced. Hence, only preprocessed data and results are being sent to the cloud for further analysis and permanent storage, which reduces latency and resloves bandwidth issues} \cite{forcan2020cloud}.

\citeauthor{7396147} investigate in how IoT systems' governance can be improved in terms of uncertainty in the system infrastructure. Uncertainty can be caused by many reasons, e.g. probe failures, network issues or human error and puts a lot of burden on the developers and operation mangagers (users) when managing runtime governance in IoT cloud systems \cite{7396147}. But it is a big challenge to include uncertainties in the development of proper governance strategies. \citeauthor{7396147} introduce the U-GovOps framework for \textit{dynamic, on-demand governance of elastic IoT cloud systems under uncertainty}. It consists of a declarative policy language that basically allows the developers to model uncertainties for their governance strategies and mechanisms that support the execution of the strategies taking the modeled uncertainties into account.

\begin{itemize}
\item Quite extensive research about on how IoT systems governance can be improved with regard to cyber security, performance and interoperability has been done. ...
\item as well as proposals on how cloud computing could support the implementation of extensive test environments is available. 
\item There is also literature about challenges of testing smart grids and how they might be solved. The focus lies mainly on the verification of smart grids' cyber and physical security and the interoperatiblity of the systems components.
\item Finally there are various papers proposing new frameworks for testing smart grids as well as testbeds.
\end{itemize}
