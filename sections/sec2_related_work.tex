\section{Related Work (rework!)}
So far no literature review covering cloud testing techniques in smart grid industry have been found. However research has been done on the topic itself.

Many authors point out the potential of cloud computing to improve efficiency, . It is relevant to mention that there is a certain consensus that hybrid cloud-fog usage, is the way to go and not using only cloud computing. (Saman Zahoor, Miodrag Forcan, ev. Nikhil Mishra (aber schlechtes Paper))

\citeauthor{7396147} investigate in how IoT systems' governance can be improved in terms of uncertainty in the system infrastructure. Uncertainty can be caused by many reasons, e.g. probe failures, network issues or human error and puts a lot of burden on the developers and operation mangagers (users) when managing runtime governance in IoT cloud systems \cite{7396147}. But it is a big challenge to include uncertainties in the development of proper governance strategies. \citeauthor{7396147} introduce the U-GovOps framework for \textit{dynamic, on-demand governance of elastic IoT cloud systems under uncertainty}. It consists of a declarative policy language that basically allows the developers to model uncertainties for their governance strategies and mechanisms that support the execution of the strategies taking the uncertainties into account.

Quite extensive research about on how IoT systems governance can be improved with regard to cyber security, performance and interoperability has been done. ...


as well as proposals on how cloud computing could support the implementation of extensive test environments is available. 

There is also literature about challenges of testing smart grids and how they might be solved. The focus lies mainly on the verification of smart grids' cyber and physical security and the interoperatiblity of the systems components.

Finally there are various papers proposing new frameworks for testing smart grids as well as testbeds.
