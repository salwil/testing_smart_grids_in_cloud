\section{Related Work}

\citeauthor{bertolino2019systematic} did a systematic review on cloud testing. They distinguish the term by two meanings: \textit{testing of the cloud (ToC)}, which refers to testing systems running in a cloud and \textit{testing in the cloud (TiC)} which refers to leveraging cloud technologies for testing. They find that the cloud \textit{offers the possibility to develop and maintain costly test infrastructures and to leverage on-demand scalable resources for configuration (by using cloud virtualization) and performance (by means of cloud elasticity) testing.}

Many authors investigated on IoT cloud computing and point out its potential to improve efficiency and reduce costs in IoT because computing resources can be flexibly scaled and virtualized. \citeauthor{laghari2021review} in their \textit{Review and State of Art of Internet of Things} point out the tight interconnection between cloud and IoT and they emphasize advantages that the intermingling of IoT and cloud have. However they also outline risks of the cloud IoT, like data ownership, potential crashes and latency.

In context of smart grids, the term hybrid-cloud usage appears very often and seems to be clearly favoured over "cloud-only" approaches for several reasons. \citeauthor{talaat2020hybrid} for example suggest to integrate data processing hardware devices to a private cloud to overcome security issues in the public cloud. \citeauthor{zahoor2018cloudmanag} recommend cloud-fog-based smart grid model for efficient resource management \cite{zahoor2018cloudmanag} and utilization \cite{zahoor2018cloudutil}. \citeauthor{zahoor2018cloudmanag} explain how the amount of transmitted data and data transmission time can be significantly reduced because the fog approach reduces the amounts of data transmitted to the cloud by performing decentralized data processing. It means that some of the processing units and storage is placed closer to where data are being sourced from. This is how fog computing has a positive impact on latency reduction and the resolvement of bandwidth issues \cite{forcan2020cloud}. It can furthermore address security issues and certain regulations (e.g. in relation to data ownership).

\citeauthor{7396147} investigate in how IoT systems' governance can be improved in terms of uncertainty in the system infrastructure. Uncertainty can be caused by many reasons, e.g. probe failures, network issues or human error and puts a lot of burden on the developers and operation mangagers (users) when managing runtime governance in IoT cloud systems \cite{7396147}. But it is a big challenge to include uncertainties in the development of proper governance strategies. \citeauthor{7396147} introduce the U-GovOps framework for \textit{dynamic, on-demand governance of elastic IoT cloud systems under uncertainty}. It consists of a declarative policy language that basically allows the developers to model uncertainties for their governance strategies and mechanisms that support the execution of the strategies taking the modeled uncertainties into account.

\begin{itemize}
\item energy consumption cost reduction thanks to smart grids (\citeauthor{bornhoft2013simulation})
\item Greencity: \citeauthor{baumgartner2020monitoring}
\item Quite extensive research on how IoT systems governance can be improved with regard to cyber security, performance and interoperability has been done. ...
\item proposals on how cloud computing could support the implementation of extensive test environments
\end{itemize}

Despite this considerable amount of literature and research about cloud computing and testing in smart grids, so far there seems to be no review available giving an overview on this.
