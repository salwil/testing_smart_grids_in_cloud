\section{Introduction}
In context of the energy transition and by being a substantial part of smart cities, evolving all around the globe, smart grids have gained increased focus in information technology and electrical engineering research during the last years. Smart grids complement traditional power grids with the application of communication and computational techniques \cite{talaat2020hybrid} i.e. with the incorporation of communication networks, intelligent automation, advanced sensors, and information technologies (\cite{smadi2021comprehensive}). By automatically monitoring, controlling and steering the generation, delivery, and consumption of electricity, the performance of the power grid in terms of reliability, efficiency and resilience can be optimized significantly and costs on the customer side can be reduced. While smart grids open doors to unprecedented possibilities, like many other technological achievements, they have their downsides at the same time. Being part of the highly critical infrastructure of electricity supply, they are exceptionally exposed and vulnerable towards malicious attacks. Furthermore they face high-performance requirements in order to fulfill real-time data processing and seamlessly integrated components ensuring flawless interoperability of the system. Successful cyber attacks or misbehavior due to badly performing systems can cause huge damage to institutions and humans that depend on this infrastructure.

Reliable security and protection layers and perfect interoperability of software and hardware are inevitable to mitigate these risks. Such systems require a high amount of thorough testing, especially with focus on their vulnerabilities. However conventional testing technologies reach their limits when facing the combination of heterogeneous and co-existing smart grids. \citeauthor{smadi2021comprehensive} stress fidelity to be one of the major factors limiting the effectiveness of existing testbeds because they do not implement sufficient interoperability, by simulating mainly the software and neglecting the physical system parts. Furthermore the physical and cyber layer lack of flexibility and are expensive to configure and finally the equipment is insufficiently diverse and heterogeneous, e.g. by assembling only devices from one vendor or supplier\cite{smadi2021comprehensive}. Elaborate test environments can be key to addressing the weaknesses of smart grids and cloud testing can make extended testing feasible by providing highly scalable test environments and resources. In general, testing should not be done on the real power system, because the deployment and usage of poductive software and hardware for testing purposes is far too expensive. Additionally, simulating disruptive actions like cyber attacks can damage a system considerably - another reason why the employment of testbeds and simulators that mimic productive environments and their components is highly recommended.

This present review study encompasses scientific papers containing research about how safety, interoperability and efficiency of IoT devices - especially smart meters -, and communication networks and data management systems in smart grids, can be tested and how cloud testing might contribute in solving some of the above mentioned problems.
