\section{Research Methodology}
The research procedure consisted of three phases: review planning, conduction and reporting the results. This process was inspired by the recommendation of the guidelines in Kitchenham et. Al's \textit{Procedures for Performing Systematic Reviews} \cite{kitchenham2004procedures}. Even though this is not a \textit{systematic} review, the fundamental principle of the approach is still adequate and useful.

\subsection{Review planning}
The research question is about how cloud testing can be employed to test IoT systems like smart grids comprehensively. It gives an overview of the most important non-functional requirements, like cyber security, efficiency, reliability, and interoperability. It summarizes research findings about how cloud computing helps to solve but also aggravates some of these critical aspects.

To get a solid overview of the context, literature about vulnerabilities of smart grids and in general about risks posed in IoT development was examined. Smart grids are actually nothing else than a type of IoT ecosystem. They utilize sensors that collect data, streaming it on a central platform, which itself, processes and stores the data and implements APIs for devices and applications to enable interaction with the system. 

Next, it was investigated, if and how cloud replicas can help preventing system lacks in one of the above mentioned risk aspects. The section IV provides an overview of findings and discusses core statements regarding chances and difficulties as well as existing or proposed solutions.

\subsection{Review conduction}

The search for the literature review included manual document retrieval from three popular web libraries: IEEE eXplore, Google Scholar, and ACM Digital Library. The documents have been retrieved by using a search term combined of the keywords \textit{smart grid} or \textit{iot}, \textit {cloud}, and \textit{testing} or \textit{simulation}. The results were restricted to the years of publication from 2018 to 2023. The terms \textit{smart grid} and \textit{IoT} were treated as synonymous correspondants during the search and selection procedure, because it turned out that many findings for general IoT testing apply for smart grid testing. Equally, \textit{testing} and \textit{simulation} were treated according to the principle of synonymy. In the electrical engineering domain the term \textit{simulation} seems to be widespread and some papers use it more often than the term \textit{testing}.

From all documents retrieved by the search, those that met suitability criteria for the topic were selected in top down manner. Suitability was assessed in a two-step approach. Documents were directly selected, if they addressed all of the three above mentioned topics (\textit{smart grid} or \textit{iot}, \textit {cloud}, \textit{testing} or \textit{simulation}) in their title or abstract. If they covered at least two in title or abstract, they were pre-selected and then scanned for occurrences of the third missing keyword. For example the content of a document, with the title \textit{Cloud-Fog-based approach for Smart Grid monitoring} was scanned, if it also covered the term \textit{testing} or \textit{simulation} in the text. 

Finally, manual backward snowballing iterations have been done on research papers that cited other papers in relation to cloud testing solutions for smart grids or IoT  with publication date from 2018 to 2023.

\subsection{Terminology}
\citeauthor{bertolino2019systematic} conducted a \textit{systematic review on cloud testing} and they distinguish the term \textit{cloud testing} by two meanings: \textit{testing of the cloud (ToC)}, which refers to testing systems running in a cloud and \textit{testing in the cloud (TiC)} which refers to leveraging cloud technologies for testing. They find that the cloud \textit{offers the possibility to develop and maintain costly test infrastructures and to leverage on-demand scalable resources for configuration (by using cloud virtualization) and performance (by means of cloud elasticity) testing.} This literature review did not have a specific focus on one of the definitions. Both shapes of cloud testing were considered, as well as their intersection, which is called \textit{testing of the cloud in the cloud (ToiC)} by \citeauthor{bertolino2019systematic}.



