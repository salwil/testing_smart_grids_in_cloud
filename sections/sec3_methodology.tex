\section{Research Methodology}
The research procedure consisted of three phases: review planning, conduction and reporting the results. This process was inspired by the recommendation of the guidelines in Kitchenham et. Al's \textit{Procedures for Performing Systematic Reviews}\cite{kitchenham2004procedures}. Even though this is not a \textit{systematic} review, the fundamental principle of the approach is still adequate and useful.

\subsection{Review planning}
The research question is about how cloud testing can support testing of software in smart grids in terms of the non-functional requirements of cyber security and performance. Firstly, to get a solid overview of the context, literature about vulnerarbilities of smart grids and in general about risks in IoT development was examined. Furthermore it was examined if and how cloud testing and cloud testbeds come into play to address these aspects. Not all of this literature was also considered for the actual review.

\subsection{Review conduction}

The search for the literature review included manual document retrieval from three popular web libraries: IEEE eXplore, Google Scholar, and ACM Digital Library. The documents have been retrieved by using the search term \textit{[smart grid / IoT] cloud [[penetration / performance / - ] testing / simulation]} and were restricted to the years of publication from 2018 to 2023. The terms \textit{smart grid} and \textit{IoT} were treated as synonymous correspondants during the search and selection procedure, because it turned out that some aspects were for general IoT software testing in literature but might suit well for smart grid devices, even if this was not explicitely mentioned / intended? by the authors.

From all documents retrieved by the search, XX were selected based on suitability criteria for the topic in top down manner. Suitability was assessed in a two-step approach. Documents were pre-selected, if they addressed at least two topics of the following: smart grid or IoT, testing, cloud computing. If they covered all of them, they were directly selected. All the pre-selected documents were then scanned for occurrences of the third missing keyword. For example the content of a document, with the title \textit{Cloud-Fog-based approach for Smart Grid monitoring} was scanned, if it also covered the term (software) testing somewehere. Finally, backward snowballing iterations have been done on the five most relevant papers, which added another XX papers to the existing review collection. The five most relevant papers were selected based on how extensively they discuss the topic of testing cloud solutions for smart grids.

