\section{Research Methodology}
The research procedure consisted of three phases: review planning, conduction and reporting the results. This process was inspired by the recommendation of the guidelines in Kitchenham et. Al's \textit{Procedures for Performing Systematic Reviews} \cite{kitchenham2004procedures}. Even though this is not a \textit{systematic} review, the fundamental principle of the approach is still adequate and useful.

\subsection{Review planning}
The research question is about how cloud testing can be employed to test comprehensive IoT systems like smart grids. It gives an overview of the most important non-functional requirements, like cyber security, efficiency, reliability, and interoperability. It summarizes research findings about how cloud computing helps to solve but also aggravates some of these critical aspects. To get a solid overview of the context, literature about vulnerabilities of smart grids and in general about risks posed in IoT development was examined. Smart grids are actually nothing else than a type of IoT ecosystem. They utilize sensors that collect data, streaming it on a central platform, which itself, processes and stores the data and implements APIs for devices and applications to enable interaction with the system. 

Next, it was investigated if and how cloud testing, cloud testbeds or co-simulations can help preventing system lacks in one of the above mentioned risk aspects. The section IV provides an overview of findings and discusses the most relevant statements, difficulties and existing or proposed solutions.

\subsection{Review conduction}

The search for the literature review included manual document retrieval from three popular web libraries: IEEE eXplore, Google Scholar, and ACM Digital Library. The documents have been retrieved by using the search term \textit{[smart grid / IoT] cloud [[security / performance / - ] testing / simulation]} and were restricted to the years of publication from 2018 to 2023. The terms \textit{smart grid} and \textit{IoT} were treated as synonymous correspondants during the search and selection procedure, because it turned out that many aspects that apply for general IoT software testing in literature can be applied smart grids as well. The search terms \textit{testing} and \textit{simulation} were applied according to the same principle of synonymy. Especially in the electrical engineering domain the term \textit{simulation} seems to be widespread and some papers use it more often than the term \textit{testing}.

From all documents retrieved by the search, XX were selected based on suitability criteria for the topic in top down manner. Suitability was assessed in a two-step approach. Documents were pre-selected, if they addressed at least two topics of the following: smart grid or IoT, testing, cloud computing. If they covered all of them, they were directly selected. All the pre-selected documents were then scanned for occurrences of the third missing keyword. For example the content of a document, with the title \textit{Cloud-Fog-based approach for Smart Grid monitoring} was scanned, if it also covered the term (software) testing somewehere. 

Finally, manual backward snowballing iterations have been done on research papers that cited others in context of testing cloud solutions for smart grids or IoT.

