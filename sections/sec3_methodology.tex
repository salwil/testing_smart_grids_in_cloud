\section{Research Methodology}
The research procedure consisted of three phases: review planning, conduction and reporting the results. This process bases on the recommendation of the guidelines in Kitchenham et. Al's \textit{Procedures for Performing Systematic Reviews}\cite{kitchenham2004procedures}.

\subsection{Review planning}
The research question is about cloud testing can support testing of software in smart grids in terms of the non-functional requirements of cyber security and performance. To get a good overview of the context, also literature about risks in IoT development, IoT cloud testing, cloud testbeds was examined. However, his literature was not considered for the actual review. 


\subsection{Review conduction}

The search included manual document retrieval from four popular web libraries: IEEE, Google Scholar, Xplore and ACM Digital Library. The documents have been retrieved by using the search term \textit{smart grid cloud testing} and were restricted to the publish years 2018 - 2023. From initially XX potential papers, XX were selected based on suitability criteria for the topic. Suitability was assessed in a two-step approach, by reading title and abstract and scanning the document for the terms that are relevant but didn't appear neither in title nor in abstract. For example the content of a document, with the title \textit{Cloud-Fog-based approach for Smart Grid monitoring} was scanned, if it also covered software testing somewehere. If not, it was excluded. Finally, backward snowballing iterations have been done on the six most relevant papers, which added another XX papers to the review. The six most relevant papers were selected based on how extensively they discuss the topic of testing cloud solutions for smart grids.



\subsection{Review results}

